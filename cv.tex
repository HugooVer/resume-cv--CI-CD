\documentclass[10pt,a4paper]{article}

% Encoding and language
\usepackage[utf8]{inputenc}
\usepackage[T1]{fontenc}
\usepackage{lmodern}
\usepackage[french]{babel}

% Layout settings
\usepackage[margin=1cm]{geometry}
\setlength{\parindent}{0pt}
\setlength{\parskip}{6pt}
\pagestyle{empty}

% Clickable links
\usepackage[hidelinks]{hyperref}
\usepackage{microtype}
\usepackage{enumitem}
\setlist[itemize]{left=0pt,itemsep=1pt,label=\textbullet,nosep}
\urlstyle{same}

\usepackage{array}
\usepackage{xcolor}
\definecolor{sectioncolor}{RGB}{221,72,20}

% Personal commands (Environment variables)
\newcommand{\nom}{\detokenize{${NAME}}}
\newcommand{\poste}{Stage DevOps}
\newcommand{\mail}{\detokenize{${MAIL}}}
\newcommand{\phone}{\detokenize{${PHONE}}}

% --- DYNAMIC LOGIC: Avoid double definitions ---
\ifdefined\ishtml
    % Script will provide the HTML-specific definitions
\else
    % PDF definitions (Only used during PDF compilation)
    \newcommand{\makecvheader}{
      \begin{minipage}[t]{0.6\textwidth}
        {\huge \textbf{\nom}}\\[0.2em]
        {\large \poste}\\[0.5em]
        \textit{Recherche un stage de DevOps de 6 mois (Disponible immédiatement)}
      \end{minipage}
      \hfill
      \begin{minipage}[t]{0.35\textwidth}
        \begin{tabular}{@{}ll}
          Téléphone : & \phone \\
          Email : & \href{mailto:\mail}{\mail{}} \\
          GitHub : & \href{https://github.com/HugooVer}{github.com/HugooVer} \\
          Ville : & Paris 02, France \\
        \end{tabular}
      \end{minipage}
      \vspace{0.4em}
    }

    \newcommand{\cvsection}[1]{
      \vspace{0.8em}
      {\large\bfseries\textcolor{sectioncolor}{#1}}\\[-0.5em]
      \rule{\linewidth}{0.4pt}\vspace{0.4em}
    }
\fi

\begin{document}

\makecvheader

% =========================
% Body
% =========================

\cvsection{Objectif}
Étudiant à l'école \textbf{42 Paris}, je recherche un \textbf{stage de 6 mois en DevOps}.  Passionné par l’automatisation et la culture 'Everything as Code', je souhaite mettre mes compétences en administration système et CI/CD au service d'une équipe de production

\cvsection{Projets}
\textbf{\href{https://github.com/HugooVer/my-vps}{Infrastructure Personnelle} + \href{https://github.com/HugooVer/resume-cv--CI-CD}{Déploiement Continu (VPS + CI/CD)}}.\\
\textit{Projet personnel}
\begin{itemize}
  \item Provisionnement et sécurisation d'un VPS Linux via Ansible (configuration Pare-feu Iptables, accès SSH, Docker)
  \item Déploiement d'un reverse-proxy Traefik avec gestion automatique des certificats SSL (Let's Encrypt)
  \item Automatisation intégrale du CV : Pipeline GitHub Actions pour la compilation (LaTeX/Pandoc) et le déploiement automatique sur le serveur
\end{itemize}

\textbf{\href{https://github.com/HugooVer/discord-repo-updater}{Maintenance et Packaging de Dépôt APT (Docker + Systemd)}}.\\
\textit{Projet personnel}
\begin{itemize}
  \item Développement d'un conteneur Docker pour automatiser la récupération et le packaging de logiciels tiers
  \item Mise en place d'un dépôt APT local sécurisé pour centraliser les mises à jour via apt upgrade
  \item Automatisation des tâches de maintenance système avec des unités systemd et scripts Bash
\end{itemize}

\cvsection{Compétences techniques}
\textbf{Systèmes, Administration} : Linux, Scripting Bash/Shell,Gestion de services (systemd), Administration serveur\\
\textbf{Conteneurisation} : Docker , Docker Compose , Gestion d'images (GHCR)\\
\textbf{Infrastructure as Code} : Ansible (Playbooks Roles), act, Automatisation de déploiement, Provisionnement de VPS\\
\textbf{CI/CD Qualité} : GitHub Actions, Automatisation de tests (Pytest), Linting (Ansible-lint)\\
\textbf{Réseaux Sécurité} : Traefik , Nginx, Iptables/Firewall, SSL/TLS (Let's Encrypt), Protocoles (TCP/IP, DNS, SSH, HTTP/s)\\
\textbf{Développement Outils} : Python, C/C++ , Git , VS Code

\cvsection{Formation}
\textbf{École 42 Paris} \hfill Depuis fin 2022 \\
Parcours : Développement/tronc commun \\
\textit{Compétences principales :} logique de programmation, travail en équipe, projets concrets, auto-apprentissage.

\textbf{BTS SN-IR} \hfill 2022 \\
ESME Sudria, Paris. \\
\textit Spécialité : Systèmes Numériques Option Informatique et Réseaux

\cvsection{Expériences professionnelles}
\textbf{Responsable informatique/Pédagogie, Animateur | Les Petits Atomes, Paris} \hfill Depuis 2020
\begin{itemize}
  \item Administrer et maintenir un parc informatique de Raspberry Pi dédié à l'apprentissage
  \item Concevoir et animer des ateliers techniques (robotique, informatique, Arduino, Science)
  \item Vulgariser des concepts informatique et scientifiques complexes auprès de divers publics.
\end{itemize}

\cvsection{Langues}
\begin{itemize}
  \item Français : maternelle
  \item Anglais : courant
\end{itemize}

\cvsection{Centres d'intérêt}
\begin{itemize}
  \item Informatique : veille technologique, auto-hébergement, open source, AI/LLM, sécurité.
  \item Intérêts personnels : impression 3D, bricolage/DIY, Arduino/ESP32, cuisine/pâtisserie.
\end{itemize}

\end{document}