\documentclass[10pt,a4paper]{article}

% Encoding and language
\usepackage[utf8]{inputenc}
\usepackage[T1]{fontenc}
\usepackage{lmodern}
\usepackage[french]{babel}

% Layout settings
\usepackage[margin=1cm]{geometry}
\setlength{\parindent}{0pt}
\setlength{\parskip}{6pt}
\pagestyle{empty}

% Clickable links
\usepackage[hidelinks]{hyperref}
\usepackage{microtype}
\usepackage{enumitem}
\setlist[itemize]{left=0pt,itemsep=1pt,label=\textbullet,nosep}
\urlstyle{same}

\usepackage{array}
\usepackage{xcolor}
\definecolor{sectioncolor}{RGB}{221,72,20}

% Personal commands (Environment variables)
\newcommand{\nom}{\detokenize{${NAME}}}
\newcommand{\poste}{Stage DevOps}
\newcommand{\mail}{\detokenize{${MAIL}}}
\newcommand{\phone}{\detokenize{${PHONE}}}

% --- DYNAMIC LOGIC: Avoid double definitions ---
\ifdefined\ishtml
    % Script will provide the HTML-specific definitions
\else
    % PDF definitions (Only used during PDF compilation)
    \newcommand{\makecvheader}{
      \begin{minipage}[t]{0.6\textwidth}
        {\huge \textbf{\nom}}\\[0.2em]
        {\large \poste}\\[0.5em]
        \textit{Recherche un stage de DevOps de 6 mois}
      \end{minipage}
      \hfill
      \begin{minipage}[t]{0.35\textwidth}
        \begin{tabular}{@{}ll}
          Téléphone : & \phone \\
          Email : & \href{mailto:\mail}{\mail{}} \\
          GitHub : & \href{https://github.com/HugooVer}{github.com/HugooVer} \\
          Ville : & Paris 02, France \\
        \end{tabular}
      \end{minipage}
      \vspace{0.4em}
    }

    \newcommand{\cvsection}[1]{
      \vspace{0.8em}
      {\large\bfseries\textcolor{sectioncolor}{#1}}\\[-0.5em]
      \rule{\linewidth}{0.4pt}\vspace{0.4em}
    }
\fi

\begin{document}

\makecvheader

% =========================
% Body
% =========================

\cvsection{Objectif}
Étudiant à l'école \textbf{42 Paris}, je recherche un \textbf{stage de 6 mois} en \textbf{DevOps} afin de mettre en pratique et d’augmenter mes compétences en administration système, automatisation et gestion d'infrastructure, et de découvrir le fonctionnement d'une équipe de production.

\cvsection{Formation}
\textbf{École 42 Paris} \hfill Depuis fin 2022 \\
Parcours : Développement/tronc commun \\
\textit{Compétences principales :} logique de programmation, travail en équipe, projets concrets, auto-apprentissage.

\textbf{BTS SN-IR} \hfill 2022 \\
ESME Sudria, Paris. \\
\textit Spécialité : Systèmes Numériques Option Informatique et Réseaux

\cvsection{Compétences techniques}
\textbf{Systèmes} : Linux (OS principal), lignes de commande (bash/sh), gestion des services, des utilisateurs et des droits.\\
\textbf{Réseau} : Notions TCP/IP, SSH, DNS, HTTP/s.\\
\textbf{DevOps} : Docker, Docker Compose, scripts d'automatisation, Ansible, CI/CD, GitHub Actions.\\
\textbf{Développement} : Python, Shell, C/C++.\\
\textbf{Outils} : Git, VS Code.

\cvsection{Projets}
\textbf{\href{https://github.com/HugooVer/discord-repo-updater}{Packaging et mise à jour d’un logiciel tiers via dépôt APT local (Docker + systemd)}}.\\
\textit{Projet personnel}
\begin{itemize}
  \item Création d’un dépôt APT local alimenté par un conteneur Docker.
  \item Automatisation avec systemd et scripts Bash.
  \item Documentation d’usage et bonnes pratiques.
\end{itemize}

\textbf{\href{https://github.com/HugooVer/CI_CD_Explorer}{Découverte CI/CD}}.\\
\textit{Projet personnel}
\begin{itemize}
  \item Développement d’une petite API REST en Python avec FastAPI et tests automatisés via pytest
  \item Mise en place d’une chaîne CI GitHub Actions (installation des dépendances, exécution des tests sur chaque push/pull request)
  \item Implémentation du CD via conteneur Docker et publication d’images versionnées/tagées sur GitHub Container Registry
\end{itemize}

\cvsection{Expériences professionnelles}
\textbf{Responsable informatique/Responsable Pédagogie/Animateur} \hfill Depuis 2020 \\
Les Petits Atomes, Paris.
\begin{itemize}
  \item Création et maintien d'un parc de Raspberry Pi à usage d’ordinateur
  \item Création d'animations informatiques et scientifiques
  \item Animation d’ateliers informatiques, robotique et scientifiques
\end{itemize}

\cvsection{Langues}
\begin{itemize}
  \item Français : langue maternelle
  \item Anglais : courant
\end{itemize}

\cvsection{Centres d'intérêt}
\begin{itemize}
  \item Informatique : veille technologique, auto-hébergement, open source, AI/LLM, sécurité.
  \item Intérêts personnels : impression 3D, bricolage, DIY, Arduino/ESP32, cuisine.
\end{itemize}

\end{document}