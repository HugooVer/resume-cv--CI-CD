\documentclass[10pt,a4paper]{article}

% Encodage et langue
\usepackage[utf8]{inputenc}
\usepackage[T1]{fontenc}
\usepackage{lmodern} % use scalable Latin Modern fonts to enable microtype features
\usepackage[french]{babel}

% Mise en page
\usepackage[margin=1cm]{geometry}
\setlength{\parindent}{0pt}
\setlength{\parskip}{2pt}
\pagestyle{empty}
% \linespread{0.96} % valeur légère pour ne pas que ça fasse "écrasé"

% Liens cliquables
\usepackage[hidelinks]{hyperref}
\usepackage{microtype} % améliore la typographie
\usepackage{enumitem} % pour contrôler les listes
\setlist[itemize]{left=0pt,itemsep=0pt,topsep=0pt,parsep=0pt,label=\textbullet,nosep}
\urlstyle{same} % affiche les url dans la même police
\hypersetup{
  pdfauthor={HugooVer},
  pdflang={fr},
  pdftitle={CV - HugooVer - Stage DevOps},
  pdfkeywords={CV, DevOps, Docker, CI/CD, Linux},
  pdfsubject={Recherche stage DevOps 6 mois}
}

% Pour mieux aligner les infos
\usepackage{array}

% Couleurs
\usepackage{xcolor}
\definecolor{sectioncolor}{RGB}{221,72,20}

% Commandes perso
\newcommand{\nom}{\detokenize{${NAME}}}
\newcommand{\poste}{Stage DevOps}
\newcommand{\mail}{\detokenize{${MAIL}}}
\newcommand{\phone}{\detokenize{${PHONE}}}


\newcommand{\cvsection}[1]{%
\vspace{0.8em}
{\large\bfseries\textcolor{sectioncolor}{#1}}\\[-0.5em]
\rule{\linewidth}{0.4pt}\vspace{-0.2em}
}

\begin{document}

% =========================
% En-tête
% =========================
\begin{minipage}[t]{0.6\textwidth}
  {\huge \textbf{\nom}}\\[0.2em]
  {\large \poste}\\[0.5em]
  \textit{Recherche un stage de DevOps de 6 mois}
\end{minipage}
\hfill
\begin{minipage}[t]{0.35\textwidth}
  % Remplace les valeurs ci-dessous
  \begin{tabular}{@{}ll}
    Téléphone : & \phone \\
    Email : & \href{mailto:\mail}{\mail{}} \\
    GitHub : & \href{https://github.com/HugooVer}{github.com/HugooVer} \\
    Ville : & Paris 02, France \\
  \end{tabular}
\end{minipage}

\vspace{0.4em}

% =========================
% Objectif
% =========================
\cvsection{Objectif}

Étudiant à l'école \textbf{42 Paris}, je recherche un \textbf{stage de 6 mois} en \textbf{DevOps} afin de mettre en pratique et d’augmenter mes compétences en administration système, automatisation et gestion d'infrastructure, et de découvrir le fonctionnement d'une équipe de production.

% =========================
% Formation
% =========================
\cvsection{Formation}

\textbf{École 42 Paris} \hfill Depuis fin 2022 \\
Parcours : Développement/tronc commun \\
\textit{Compétences principales :} logique de programmation, travail en équipe, projets concrets, auto-apprentissage

\vspace{0.4em}

\textbf{BTS SN-IR} \hfill 2022 \\
ESME Sudria, Paris
\begin{itemize}[left=0pt, label=\textbullet, nosep]
  \item Spécialité : Systèmes Numériques Option Informatique et Réseaux
\end{itemize}

% =========================
% Compétences techniques
% =========================
\cvsection{Compétences techniques}

\textbf{Systèmes} : Linux (OS principal), lignes de commande (bash/sh), gestion des services, des utilisateurs et des droits\\
\textbf{Réseau} : Notions TCP/IP, SSH, DNS, HTTP/s\\
\textbf{DevOps} : Docker, scripts d'automatisation, CI/CD\\
\textbf{Développement} : Python, Shell, C/C++\\
\textbf{Outils} : Git, VS Code

% =========================
% Projets (école / perso)
% =========================
\cvsection{Projets}

\textbf{\href{https://github.com/HugooVer/discord-repo-updater}{Packaging et mise à jour d’un logiciel tiers via dépôt APT local (Docker + systemd)}}\\
\textit{Projet personnel}
\begin{itemize}[left=0pt, label=\textbullet, nosep]
  \item Création d’un dépôt APT local alimenté par un conteneur Docker (récupération du paquet .deb officiel)
  \item Automatisation avec systemd et scripts Bash ; mise à niveau via apt upgrade comme un paquet standard
  \item Documentation d’usage et bonnes pratiques (gestion des permissions, sécurité, flux apt update/apt upgrade)
\end{itemize}

\vspace{0.4em}

\textbf{\href{https://github.com/HugooVer/CI_CD_Explorer}{Découverte CI/CD}}\\
\textit{Projet personnel}
\begin{itemize}[left=0pt, label=\textbullet, nosep]
  \item Développement d’une petite API REST en Python avec FastAPI et tests automatisés via pytest
  \item Mise en place d’une chaîne CI GitHub Actions (installation des dépendances, exécution des tests sur chaque push/pull request)
  \item Implémentation du CD via conteneur Docker et publication d’images versionnées/tagées sur GitHub Container Registry
\end{itemize}

% =========================
% Expériences
% =========================

\cvsection{Expériences professionnelles}

\textbf{Responsable informatique/Responsable Pédagogie/Animateur} \hfill Depuis 2020 \\
Les Petits Atomes, Paris
\begin{itemize}[left=0pt, label=\textbullet, nosep]
  \item Création et maintien d'un parc de Raspberry Pi à usage d'ordinateur pour des enfants
  \item Création d'animations informatiques et scientifiques pour des enfants
  \item Animation d'ateliers informatiques, robotique et scientifiques
\end{itemize}

% =========================
% Langues
% =========================
\cvsection{Langues}

Français : langue maternelle \\
Anglais : courant

% =========================
% Centres d'intérêt
% =========================
\cvsection{Centres d'intérêt}

\begin{itemize}[left=0pt, label=\textbullet, nosep]
  \item Informatique : veille technologique, auto-hébergement, open source, vie privée, AI/LLM, sécurité informatique
  \item Intérêts personnels : impression 3D/gravure et découpe laser, bricolage, DIY, Arduino/ESP32, cuisine/pâtisserie
\end{itemize}

\end{document}
